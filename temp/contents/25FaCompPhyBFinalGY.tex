\documentclass[11pt]{exam-zh} 
\newcommand{\udline}{\underline{\qquad $\blacktriangle$ \qquad}}
\usepackage{amsmath}
\usepackage{geometry}
\usepackage{fancyhdr}
\usepackage{physics}
\usepackage{wrapfig}
\usepackage{dsfont}
\usepackage{multicol}
\usepackage{multirow}
\pagestyle{fancy}


\geometry{papersize={21cm,29.7cm}}
\geometry{left=2cm,right=2cm,top=3cm,bottom=4cm}
\title{中国科学技术大学 \quad 2026 年 1 月 17 日 8:30 - 10:30}
\subject{2025 秋计算物理 B 期末考试}


\newcommand{\mi}{\mathrm i}%虚数单位
\newcommand{\me}{\mathrm e}%自然底数
\newcommand{\pp}{\partial }

\begin{document}
    \maketitle
    \begin{notice}
    \item 本次考试为半开卷考试.
    \end{notice}

    \section{蒙特卡洛抽样}
\begin{question}
考虑宽度为 $L=1$ 的一维无限深方势阱($0 \leq x \leq 1$). 系统处于混合态,其中有 $0.8$ 的概率位于基态 $\psi_1(x)$,有 $0.2$ 的概率位于第一激发态 $\psi_2(x)$. 
\begin{enumerate}[label=(\arabic*)]
    \item 写出该混合态的位置概率密度函数 $P(x)$. 
    \item 阐述如何利用\textbf{舍选法(Rejection Method)}对该系统的位置 $x$ 进行抽样(需给出具体的比较函数和步骤). 
    \item 说明如何基于生成的随机样本点,获得观测量 $F(x)=x^2$ 的期望值. 
\end{enumerate}


\end{question}

\vspace{8em}
\section{随机游走与线性方程组}
\begin{question}
    设 $M$ 为 100 阶三对角方阵,其主对角元为 $1-a$,次对角元均为 $-b$. 已知参数满足 $a+2b=1$ ($a, b > 0$). 

请设计一个基于\textbf{随机游走(Random Walk)}的算法,求解线性方程组 $MX=Y$. 请明确定义随机游走的转移概率、边界处理以及解的统计估算方法. 
\end{question}

\vspace{8em}
\section{有限差分法}
\begin{question}
    在极坐标系 $(r, \theta)$ 下,考虑定义在圆环区域上的 Laplace 方程:
$$
\nabla^2 u = \frac{1}{r} \frac{\partial}{\partial r} \left( r \frac{\partial u}{\partial r} \right) + \frac{1}{r^2} \frac{\partial^2 u}{\partial \theta^2} = 0
$$
请给出该方程的有限差分离散格式(中心差分),并说明边界条件的处理方式. 
\end{question}

\vspace{8em}
\section{线性代数迭代法}
\begin{question}
    考虑一个三阶线性方程组 $Ax=b$(注:此处回忆版未给出具体数值,考试时请代入试卷给定的具体矩阵 $A$ 和向量 $b$). 

要求使用 \textbf{Gauss-Seidel 迭代法},给定初值 $x^{(0)}$,请手动计算前两步迭代结果 $x^{(1)}$ 和 $x^{(2)}$. 
\end{question}


\vspace{8em}
\section{网格剖分}
\begin{question}
    给定平面上的点集,包含单位圆边界上的8个点:
$$
P_n = \left(\cos\frac{n\pi}{4}, \sin\frac{n\pi}{4}\right), \quad n=0, 1, \dots, 7
$$
以及圆内部的2个点:$Q_1 = (0.1, 0)$,$Q_2 = (-0.1, 0)$. 
\begin{enumerate}[label=(\arabic*)]
    \item 请画出或描述一种合适的 \textbf{Delaunay 三角剖分}方案. 
    \item 阐述在计算机程序中应如何设计数据结构来有效地存储此剖分(包括节点信息、单元连接关系等). 
\end{enumerate}
\end{question}


\vspace{8em}
\section{一维有限元}
\begin{question}
    考虑定义在区间 $[0, 1]$ 上的边值问题:
$$
y'' + \pi^2 y = 0
$$
边界条件为:$y(0) = 1/2$, $y(1) + 0.2y'(1) = -0.67$. 
\begin{enumerate}[label=(\arabic*)]
    \item 给出该微分方程对应的泛函 $J[y]$,并证明该泛函取极值的条件即为原微分方程成立(需包含自然边界条件的推导). 
    \item 选取节点 $x_i \in \{0, 0.2, 0.5, 1\}$ 进行离散,请列出求解该问题的有限元线性方程组(刚度矩阵和载荷向量的形式,不必算出最终数值解). 
\end{enumerate}
\end{question}

\vspace{8em}
\section{分子动力学与自旋模型}
\begin{question}
    考虑类似于 Ising 模型的自旋系统,哈密顿量 $H$ 仅包含相邻自旋的相互作用. 自旋的时间演化遵从如下方程(Landau-Lifshitz-Gilbert 方程形式):
$$
\frac{d\mathbf{S}_i}{dt} = \alpha (\mathbf{S}_i \times \mathbf{B}_i) + \beta \left( \mathbf{S}_i \times \frac{d\mathbf{S}_i}{dt} \right)
$$
其中有效场 $\mathbf{B}_i = -\frac{\partial H}{\partial \mathbf{S}_i}$. 

请结合\textbf{分子动力学(Molecular Dynamics)}的思想,设计一套数值算法求解一维链状系统中的平衡态构型. 
\end{question}


\vspace{8em}
\section{最小像力约定与邻近搜索}
\begin{question}
    在二维正方形区域 $\Omega = [0, 1] \times [0, 1]$ 中采用周期性边界条件. 设有五个粒子,坐标分别为:
\begin{center}
    1: $(0.1, 0.1)$ \quad 2: $(0.2, 0.3)$ \quad 3: $(0.3, 0.9)$ \quad 4: $(0.4, 0.4)$ \quad 5: $(0.8, 0.8)$
\end{center}
设定相互作用的截断距离为 $r_c = 0.4$. 根据\textbf{最小像力约定(Minimum Image Convention)}:
\begin{enumerate}[label=(\arabic*)]
    \item 判断哪些粒子(或其像)会对粒子 1 产生作用力. 
    \item 给出这些对粒子 1 施加力的源粒子(或其像)在计算力时使用的实际坐标. 
\end{enumerate}
\end{question}




\end{document}